\documentclass[a4paper,12pt,danish]{article}
\usepackage[T1]{fontenc}
\usepackage{babel}

\title{\bf F-klubbens vedt{\ae}gter}
\author{Som vedtaget på ordin{\ae}r generalforsamling}
\date{Sidst ændret d. \today}
\begin{document}
\maketitle

%\tableofcontents

\newcounter{fpara}



\begin{list}
{\S \arabic{fpara}}{\usecounter{fpara}}
\section{Form{\aa}l og medlemsskab}
\item Foreningens navn er F-klubben. Foreningens hjemsted er Aalborg
  kommune. Foreningens regnskabs{\aa}r er fra 1.7.\ til 30.6.
  
\item Foreningens form{\aa}l er at fremme det sociale milj{\o},
  prim{\ae}rt for ansatte og studerende tilknyttet de IT og
  naturvidenskabelige uddannelser ved Aalborg Universitet. Foreningen
  skal skabe grobund for sociale aktiviteter i det daglige, s{\aa}vel
  som arrangere s{\ae}rlige aktiviteter, der kan g{\o}re det sjovere at
  v{\ae}re ved universitetet og knytte b{\aa}nd mellem ansatte og
  studerende, mellem fag og mellem semestre.

\item Enhver med tilknytning til de IT og naturvidenskabelige uddannelser ved Aalborg Universitet, kan blive medlem.

\item Medlemsskab af F-klubben er livsvarigt, mod indbetaling af indmeldelsesgebyret.
  St{\o}rrelsen af gebyret fasts{\ae}ttes p{\aa} F-klubbens ordin{\ae}re generalforsamling.
  
\item Et F-klub-medlem, der har ydet en s{\ae}rlig indsats for
  F-klubben, kan ved en begrundet beslutning truffet af F-klubbens
  ordin{\ae}re generalforsamling nomineres til {\ae}resmedlem. Et
  {\ae}resmedlemsskab er livsvarigt og giver ret til et {\ae}reskrus. For
  at et medlem kan blive fuldgyldigt {\ae}resmedlem, skal {\ae}resmedlemskabet
	godkendes p{\aa} den f{\o}lgende ordin{\ae}re
  generalforsamling.
  
\item Et F-klub-medlem, der har udvist en for F-klubben yderst
  skadelig eller pinlig adf{\ae}rd, kan ved en begrundet beslutning
  truffet af F-klubbens ordin{\ae}re generalforsamling tildeles en
  n{\ae}se.  N{\ae}ser kan kun oph{\ae}ves af F-klubbens ordin{\ae}re
  generalforsamling.

\section{F-klubbens generalforsamling}

\item \label{par-gen} F-klubbens ordin{\ae}re generalforsamling
  s{\o}ges afholdt hvert {\aa}r i september m{\aa}ned. Indkaldelse sker med 14
  dages varsel ved udsendelse af elektronisk post til alle aktive
  medlemmer. Bestyrelsen kan i tilf{\ae}lde af force maheure uds{\ae}tte 
  generalforsamlingen. Generalforsamligen skal i s{\aa} fald indkalses 
  snarest efter det igen er muligt. Indkaldelsen skal indeholde et forslag til
  dagsorden, udarbejdet af bestyrelsen. Dagsorden ved en ordin{\ae}r
  generalforsamling b{\o}r indeholde f{\o}lgende punkter, og som minimum
punkterne markeret ved (*):

\begin{enumerate}
\item (*) Godkendelse af dagsorden
\item (*) Valg af dirigent og referent
\item (*) Bestyrelsens beretning om foreningens virksomhed
\item Initivudvalgets beretning
\item Treo'ens beretning
\item (*) Indkomne forslag
\item (*) Freml{\ae}ggelse og godkendelse af {\aa}rsregnskab
\item (*) Fasts{\ae}ttelse af indmeldelsesgebyret
\item (*) Valg af kasserer
\item (*) Valg af fire {\o}vrige bestyrelsesmedlemmer
\item Valg til treo
\item Valg til initivudvalg
\item Valg til F-{\ae}drer{\aa}d
\item (*) Valg af revisor
\item Uddeling af n{\ae}ser
\item Udn{\ae}vnelse af {\ae}resmedlemmer
\item Rehabilitering af n{\ae}semodtagere
\item Fasts{\ae}ttelse af dato for julefrokost
\item (*) Eventuelt
\end{enumerate}

Forslag, der {\o}nskes behandlet p{\aa} generalforsamlingen, skal
v{\ae}re bestyrelsen i h{\ae}nde senest en uge f{\o}r m{\o}det finder
  sted.
  
\item Ekstraordin{\ae}r generalforsamling kan indkaldes, n{\aa}r
  bestyrelsen finder det p{\aa}kr{\ae}vet, eller n{\aa}r mindst 20 medlemmer 
  forlanger det. I sidstn{\ae}vnte tilf{\ae}lde skal generalforsamlingen afholdes 
	senest fire uger efter, at foresp{\o}rgslen er kommet bestyrtelsen i
	h{\ae}nde.
  En ekstraordin{\ae}r generalforsamling indkaldes efter samme regler som i \S
  \ref{par-gen}. Dog er det ikke muligt for bestyrrelsen af uds{\ae}tte en ekstraordinær
  generalforsamling. Dagsordenen fasts{\ae}ttes af bestyrelsen, dog kan et flertal 
  af de fremm{\o}dte kr{\ae}ve valg til bestyrelsen. 
  
\item \label{gen-regler} Generalforsamlingen tr{\ae}ffer sine beslutninger med 
  simpelt flertal blandt de fremm{\o}dte, med mindre andet er n{\ae}vnt i vedt{\ae}gterne.
  Dirigenten afg{\o}r, hvorledes afstemningen skal foreg{\aa}.
  Ved stemmelighed forkastes forslaget. 
  Ved stemmelighed i forbindelse med valg til tillidsposter afg{\o}res valget 
  ved et spil t{\ae}nkeboks efter F-klub-regler (disse regler er i bestyrelsens besiddelse.)

\section{Bestyrelsen og de faste udvalg}

\item Treo'en best{\aa}r ud over kassereren (se \S \ref{par-best} og
  \ref{par-kas}) af mindst tre medlemmer. Treo'en har det daglige
  ansvar for F-orsyninger til F-klubbens kaffestuer (F-lydende og F-ast f{\o}de),
  s{\aa}vel materielt som {\o}konomisk.
  
\item Initivudvalget, der s{\o}ges sammensat af et bredt
  repr{\ae}sentantskab af F-klubbens aktive medlemmer, har til opgave at
  arrangere sociale aktiviteter s{\aa}som F-ester, ud-F-lugter,
  F-{\ae}llessport, med mere.
  
\item \label{par-best} F-klubben ledes af bestyrelsen, der best{\aa}r
  af kasseren og fire {\o}vrige medlemmer, valgt ved
  generalforsamlingen.  Bestyrelsens {\o}vrige medlemmer konstituerer sig
  selv med formand, n{\ae}stformand, sekret{\ae}r og et menigt medlem. Intet
  bestyrelsesmedlem m{\aa} have to hverv.

\item \label{par-kas} Kassereren v{\ae}lges separat p{\aa}
  generalforsamlingen.  Kassereren er f{\o}dt medlem af Treoen og
  bestyrelsen og har intet arbejdsm{\ae}ssigt ansvar i Treoen udover
  selve kass\'{e}rer-hvervet.

\item Bestyrelsen har det overordnede ansvar for F-klubbens
  aktiviteter. Bestyrelsen har mandat til at tr{\ae}ffe beslutninger om
  F-klubbens daglige drift. S{\aa}danne beslutninger kan kr{\ae}ves
  retf{\ae}rdiggjort af ethvert af F-klubbens medlemmer p{\aa} n{\ae}stkommende
  generalforsamling.

%\item En afg{\aa}ende bestyrelse skal inden nyvalg til bestyrelsen
%  indhente kandidater til den nye bestyrelse.
  
\item Bestyrelsen fordeler selv sine arbejdsopgaver, der som minimum
  best{\aa}r af at holde kontakt med de autonome F-klub-udvalg, at
  v{\ae}re ansvarlig for faglige arrangementer, og at holde styr med
  F-klubbens IT- og papirsvirke.


\item F-{\ae}drer{\aa}det best{\aa}r af et antal
  personer der har v{\ae}ret medlem af F-klubben i mindst tre {\aa}r.
  Medlemmer af F-{\ae}drer{\aa}det m{\aa} gerne v{\ae}re F-eminine.
  F-{\ae}drer{\aa}det er konsulent for bestyrelsen. F-{\ae}drer{\aa}det
  har m{\o}de- og tale- men ikke stemmeret til bestyrelsesm{\o}derne.

\section{Tvist og voldgift}

\item S{\aa}fremt der opst{\aa}r uenighed mellem foreningen og et
  eller flere af dets medlemmer, og striden ikke kan bil{\ae}gges ved
  forhandling, kan sagen ikke indbringes for domstole, men skal
  afg{\o}res ved voldgift. Det samme g{\ae}lder, hvis der opst{\aa}r
  uenighed mellem to medlemmer af foreningen ang{\aa}ende
  sp{\o}rgsm{\aa}l, der vedr{\o}rer deres stilling som medlemmer af
  foreningen. 
  
  Voldgiftsretten skal best{\aa} af to voldgiftsm{\ae}nd, hvoraf hver
  af parterne v{\ae}lger \'{e}n, samt af en opmand, der v{\ae}lges af
  voldgiftsm{\ae}ndene. Kan disse ikke enes om valget af opmand,
  udpeges denne af dommeren i den retskreds, hvor foreningen har
  hjemsted. For sagens behandling g{\ae}lder reglerne i lov om
  voldgift, nr. 181 af 24. maj 1972, eller eventuelle af Folketinget
  vedtagne erstattende love.

\section{Vedt{\ae}gts{\ae}ndringer}
  
\item F-klubbens vedt{\ae}gter kan {\ae}ndres p{\aa}
  generalforsamlinger. Forslag til vedt{\ae}gts{\ae}ndringer skal
  bekendtg{\o}res senest 14 dage inden generalforsamlingen. For at en
  vedt{\ae}gts{\ae}ndring skal kunne vedtages, kr{\ae}ves et flertal
  p{\aa} $\frac{2}{3}$ ved to p{\aa} hinanden f{\o}lgende
  generalforsamlinger. Disse generalforsamlinger skal finde sted med
  mindst 2 m{\aa}neders mellemrum.
  Der kan ikke indkaldes til generalforsamling i universitetets ferieperioder.

\end{list}
\end{document}
