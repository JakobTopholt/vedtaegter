\documentclass[a4paper,12pt,danish]{article}
\usepackage[utf8]{inputenc}
\usepackage[T1]{fontenc}
\usepackage{babel}

\title{\bf F-klubbens vedtægter}
\author{Som vedtaget på ordinær generalforsamling}
\date{Sidst ændret d. \today}
\begin{document}
\maketitle

%\tableofcontents

\newcounter{fpara}



\begin{list}
{\S \arabic{fpara}}{\usecounter{fpara}}
\section{Formål og medlemsskab}
\item Foreningens navn er F-klubben. Foreningens hjemsted er Aalborg
  kommune. Foreningens regnskabsår er fra 1.7.\ til 30.6.
  
\item Foreningens formål er at fremme det sociale miljø,
  primært for ansatte og studerende tilknyttet de IT og
  naturvidenskabelige uddannelser ved Aalborg Universitet. Foreningen
  skal skabe grobund for sociale aktiviteter i det daglige, såvel
  som arrangere særlige aktiviteter, der kan gøre det sjovere at
  være ved universitetet og knytte bånd mellem ansatte og
  studerende, mellem fag og mellem semestre.

\item Enhver med tilknytning til de IT og naturvidenskabelige uddannelser ved Aalborg Universitet, kan blive medlem.

\item Medlemsskab af F-klubben er livsvarigt, mod indbetaling af indmeldelsesgebyret.
  Størrelsen af gebyret fastsættes på F-klubbens ordinære generalforsamling.
  
\item Et F-klub-medlem, der har ydet en særlig indsats for
  F-klubben, kan ved en begrundet beslutning truffet af F-klubbens
  ordinære generalforsamling nomineres til æresmedlem. Et
  æresmedlemsskab er livsvarigt og giver ret til et æreskrus. For
  at et medlem kan blive fuldgyldigt æresmedlem, skal æresmedlemskabet
	godkendes på den følgende ordinære
  generalforsamling.
  
\item Et F-klub-medlem, der har udvist en for F-klubben yderst
  skadelig eller pinlig adfærd, kan ved en begrundet beslutning
  truffet af F-klubbens ordinære generalforsamling tildeles en
  næse.  Næser kan kun ophæves af F-klubbens ordinære
  generalforsamling.

\section{F-klubbens generalforsamling}

\item \label{par-gen} F-klubbens ordinære generalforsamling
  søges afholdt hvert år i september måned. Indkaldelse sker med 14
  dages varsel ved udsendelse af elektronisk post til alle aktive
  medlemmer. Bestyrelsen kan i tilfælde af force maheure udsætte 
  generalforsamlingen. Generalforsamligen skal i så fald indkalses 
  snarest efter det igen er muligt. Indkaldelsen skal indeholde et forslag til
  dagsorden, udarbejdet af bestyrelsen. Dagsorden ved en ordinær
  generalforsamling bør indeholde følgende punkter, og som minimum
punkterne markeret ved (*):

\begin{enumerate}
\item (*) Godkendelse af dagsorden
\item (*) Valg af dirigent og referent
\item (*) Bestyrelsens beretning om foreningens virksomhed
\item Initivudvalgets beretning
\item Treo'ens beretning
\item (*) Indkomne forslag
\item (*) Fremlæggelse og godkendelse af årsregnskab
\item (*) Fastsættelse af indmeldelsesgebyret
\item (*) Valg af kasserer
\item (*) Valg af fire øvrige bestyrelsesmedlemmer
\item (*) I  tilfælde af en fuld bestyrelse, det vil sige en kasserer og seks øvrige bestyrelsesmedlemmer(jævnført §12) , kan op til to suppleanter til bestyrelsen vælges.
\item Valg til treo
\item Valg til initivudvalg
\item Valg til F-ædreråd
\item (*) Valg af revisor
\item Uddeling af næser
\item Udnævnelse af æresmedlemmer
\item Rehabilitering af næsemodtagere
\item Fastsættelse af dato for julefrokost
\item (*) Eventuelt
\end{enumerate}
I tilfælde af et fraværende bestyrelsesmedlem, vil en suppleant,
efter det fraværende bestyrelsesmedlemmets valg, supplere det
fraværende bestyrelsesmedlem. I tilfælde af et aftrædende
bestyrelsesmedlem vil suppleanten blive et fyldbyrdigt bestyrelsesmedlem.


Forslag, der ønskes behandlet på generalforsamlingen, skal
være bestyrelsen i hænde senest en uge før mødet finder sted.
  
\item Ekstraordinær generalforsamling kan indkaldes, når
  bestyrelsen finder det påkrævet, eller når mindst 20 medlemmer 
  forlanger det. I sidstnævnte tilfælde skal generalforsamlingen afholdes 
	senest fire uger efter, at forespørgslen er kommet bestyrtelsen i
	hænde.
  En ekstraordinær generalforsamling indkaldes efter samme regler som i \S
  \ref{par-gen}. Dog er det ikke muligt for bestyrrelsen af udsætte en ekstraordinær
  generalforsamling. Dagsordenen fastsættes af bestyrelsen, dog kan et flertal 
  af de fremmødte kræve valg til bestyrelsen. 
  
\item \label{gen-regler} Generalforsamlingen træffer sine beslutninger med 
  simpelt flertal blandt de fremmødte medlemmer, med mindre andet er nævnt i vedtægterne.
  Dirigenten afgør, hvorledes afstemningen skal foregå.
  Ved stemmelighed forkastes forslaget. 
  Ved stemmelighed i forbindelse med valg til tillidsposter afgøres valget 
  ved et spil tænkeboks efter F-klub-regler (disse regler er i bestyrelsens besiddelse.)

\section{Bestyrelsen og de frecolor-keephue: trueaste udvalg}

\item Treo'en består ud over kassereren (se \S \ref{par-best} og
  \ref{par-kas}) af mindst tre medlemmer. Treo'en har det daglige
  ansvar for F-orsyninger til F-klubbens kaffestuer (F-lydende og F-ast føde),
  såvel materielt som økonomisk.
  
\item Initivudvalget, der søges sammensat af et bredt
  repræsentantskab af F-klubbens aktive medlemmer, har til opgave at
  arrangere sociale aktiviteter såsom F-ester, ud-F-lugter,
  F-ællessport, med mere.
  
\item \label{par-best} F-klubben ledes af bestyrelsen, der består
  af kasseren og op til seks øvrige medlemmer, valgt ved
  generalforsamlingen.  Bestyrelsens øvrige medlemmer konstituerer sig
  selv med formand, næstformand, sekretær og et menigt medlem. Intet
  bestyrelsesmedlem må have to hverv.

\item \label{par-kas} Kassereren vælges separat på
  generalforsamlingen.  Kassereren er født medlem af Treoen og
  bestyrelsen og har intet arbejdsmæssigt ansvar i Treoen udover
  selve kass\'{e}rer-hvervet.

\item Bestyrelsen har det overordnede ansvar for F-klubbens
  aktiviteter. Bestyrelsen har mandat til at træffe beslutninger om
  F-klubbens daglige drift. Sådanne beslutninger kan kræves
  retfærdiggjort af ethvert af F-klubbens medlemmer på næstkommende
  generalforsamling.

%\item En afgående bestyrelse skal inden nyvalg til bestyrelsen
%  indhente kandidater til den nye bestyrelse.
  
\item Bestyrelsen fordeler selv sine arbejdsopgaver, der som minimum
  består af at holde kontakt med de autonome F-klub-udvalg, at
  være ansvarlig for faglige arrangementer, og at holde styr med
  F-klubbens IT- og papirsvirke.


\item F-orældrerådet består af et antal
  personer der har været medlem af F-klubben i mindst tre år.
  Medlemmer af F-ædrerådet må gerne være F-eminine.
  F-orældrerådet er konsulent for bestyrelsen. F-ældrerådet
  har møde- og tale- men ikke stemmeret til bestyrelsesmøderne.

\section{Tvist og voldgift}

\item Såfremt der opstår uenighed mellem foreningen og et
  eller flere af dets medlemmer, og striden ikke kan bilægges ved
  forhandling, kan sagen ikke indbringes for domstole, men skal
  afgøres ved voldgift. Det samme gælder, hvis der opstår
  uenighed mellem to medlemmer af foreningen angående
  spørgsmål, der vedrører deres stilling som medlemmer af
  foreningen. 
  
  Voldgiftsretten skal bestå af to voldgiftsmænd, hvoraf hver
  af parterne vælger \'{e}n, samt af en opmand, der vælges af
  voldgiftsmændene. Kan disse ikke enes om valget af opmand,
  udpeges denne af dommeren i den retskreds, hvor foreningen har
  hjemsted. For sagens behandling gælder reglerne i lov om
  voldgift, nr. 181 af 24. maj 1972, eller eventuelle af Folketinget
  vedtagne erstattende love.

\section{Vedtægtsændringer}
  
\item F-klubbens vedtægter kan kun ændres på
  generalforsamlinger. Forslag til vedtægtsændringer skal
  bekendtgøres senest 14 dage inden generalforsamlingen. For at en
  vedtægtsændring skal kunne vedtages, kræves et flertal
  på $\frac{2}{3}$. Der kan ikke indkaldes til generalforsamling
  i universitetets ferieperioder.

\end{list}
\end{document}
